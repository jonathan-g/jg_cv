%
% !TeX root = Gilligan_CV_2015.tex
%
\begin{enumerate}
%
    \item
    \textdagger\
    \enquote{Drought, Risk, and Institutional Politics in the American Southwest,}
    by D.J. Hess, *C.A. Wold, *E. Hunter, *J. Nay, *S. Worland, \textbf{J. Gilligan}, and G.M. Hornberger,
    Sociological Forum, in press.
%    
    \item 
	\textdagger\
	 \enquote{Spatiotemporal Patterns of Agricultural Drought in Sri Lanka: 1881--2010,}
	    by *T. Gunda, G.M. Hornberger, and \textbf{J.M. Gilligan}, 
	    International Journal of Climatology \textbf{36}, 263--575 (2015).
        \begin{credit}
        This paper was largely Gunda's work. I provided a lot of guidance on her spatiotemporal statistical analysis and on interpreting the results of that analysis. Gunda wrote the bulk of the paper, with contributions from Hornberger and myself.
        \end{credit}
%
    \item 
	\textdagger\ 
	\enquote{Water Conservation and Hydrological Transitions in Cities in the United States,}
	    by G.M. Hornberger, D.J. Hess, and \textbf{J.M. Gilligan}, 
	    Water Resources Research \textbf{51}, 4635--4649 (2015).
	    \begin{credit}
	    This project was conceived and designed equally by Hornberger, Hess, and myself. Hornberger and Hess did the bulk of research, analysis, and writing on this paper. I participated in writing and contributed a small amount of research and statistical analysis.
	    \end{credit}
%
    \item 
	\enquote{Beyond Gridlock,}
	    by M.P. Vandenbergh and \textbf{J.M. Gilligan}, 
	    Columbia Journal of Environmental Law
        \textbf{40}, 217--303 (2015).
        \begin{credit}
        This paper was roughly equally written by Vandenbergh and myself. Vandenbergh contributed expertise on legislative history and policymaking. I contributed quantitative analysis of greenhouse gas emissions, risk analysis, and risk management. We contributed equally to material on the role of individual behavior.
        \end{credit}
%
    \item 
	\textdagger\ 
	\enquote{Flood risk of natural and embanked landscapes on the Ganges-Brahmaputra tidal delta plain,}
	    by *L.W. Auerbach, S.L. Goodbred, Jr., D.R. Mondal, C.A. Wilson, *K.R. Ahmed, K. Roy, M.S. Steckler, C. Small, \textbf{J.M. Gilligan}, and B.A. Ackerly,
	    Nature Climate Change
        \textbf{5}, 153--157 (2015).
        \begin{credit}
        The project was conceived jointly by myself, Ackerly, and Goodbred. The bulk of this paper focused on geological measurements and analysis designed and conducted by Auerbach, Goodbred, Mondal, and Wilson. I contributed synthesis between the geological and social sciences, made with land-use in other parts of the delta, and did historical research on the evolution of scientific thinking about sediment dynamics and flood control in Bengal since the 19th century.
        \end{credit}
%
    \item 
	\enquote{Energy and Climate Change: A Climate Prediction Market,}
	    by M.P. Vandenbergh, K.E. Toner, and \textbf{J.M. Gilligan}, 
	    UCLA Law Review 
        \textbf{61}, 1962--2017 (2014).
	  \begin{credit}
	As scientific evidence has become increasingly decisive that human activity is changing the climate, the public has remained unsure: less than half of the public knows that human activity is changing the climate, and a partisan divide those who identify themselves as Democrats and those who identify as Republicans has doubled since 2001. In this paper, we explore the proposition that investment markets in predictions about future climate change, similar to the Iowa Electronic Market for predicting election outcomes, might serve as a trustworthy source of information about climate change for those who do not trust scientists. We examine the declining trust in science, the prospect that those who distrust scientists might trust investment markets, and the legal and economic issues that a proposed private climate prediction market would have to address.  Vandenbergh and Gilligan originated the idea of a climate prediction market. Vandenbergh and Toner developed the ideas: Vandenbergh researched the legal issues around prediction markets and researched prediction markets, such as IEM and InTrade. Toner researched the psychological issues around trust. Gilligan researched past proposals for climate prediction markets, bets on climate change between climate scientists and skeptics, and contributed economic analysis of possible sources of market failure in a prediction market. Vandenbergh and Toner did most of the writing, with Gilligan contributing between 10\% and 25\% of the final text. In the course of writing, all three authors conducted extensive discussions of the entirety of the paper.
	  \end{credit}
%
    \item 
	\enquote{Accounting for political feasibility in climate instrument choice,}  
	    by \textbf{J.M. Gilligan} and M.P. Vandenbergh, 
	    Virginia Environmental Law Journal
	    \textbf{32}, 1--26 (2014).
%
    \item 
	\enquote{Accounting for political feasibility in climate instrument choice,}  
	    by \textbf{J.M. Gilligan} and M.P. Vandenbergh, 
	    Virginia Environmental Law Journal
	    \textbf{32}, 1--26 (2014).
	  \begin{credit}
	  We introduce a concept we call ``political opportunity cost'' as a tool for analyzing climate-change policy. Climate policy debates have become polarized between those who favor sweeping comprehensive emissions-regulation measures, such as cap-and-trade or emissions taxes; and those who see comprehensive measures as politically unfeasible and advocate focusing on smaller measures that can feasibly be enacted in the short term. We argue that this is a false dichotomy and that climate policy requires a harmonious combination of short-term measures that can achieve modest, but meaningful deceleration of emissions growth with more ambitious long-term measures that will be necessary to stabilize greenhouse gas emissions at levels that avoid dangerous climate change. We propose a new direction for policy analysis that accounts for the opportunity costs that arise when pursuing more ambitious but politically controversial policies instead of modest but less controversial ones, and we argue that this approach could prove a way out of the paralyzing polarization between policies that deliver too little and those that deliver too late. 
	  Vandenbergh and I made equal contributions to this paper. I analyzed the consequences of delay on preventing catastrophic risk and developed the idea of political opportunity cost. 
	  Vandenbergh built on this to apply the political opportunity cost framework to analyzing policy options currently under discussion in the U.S.
	  \end{credit}
	  
%
	\item
	\enquote{Macro-risks: The challenge for rational risk regulation,}  by M.P.
	  Vandenbergh and \textbf{J.M. Gilligan}, Duke Environmental Law and Policy Forum
	  \textbf{21}, 401--431 (2011). 
	  \begin{credit}
	  Global environmental hazards, such as climate change, are fundamentally different from most environmental issues because they are irreversible on millennial timescales and have a fat tail at the high-damage end of the probability distribution that implies small but non-negligible risks of disrupting civilization itself. This both renders conventional cost-benefit analysis invalid and makes economic policy analysis vulnerable to the same errors that led to the 2007-08 meltdown of the markets for mortgage-backed securities. We propose a new approach to evaluating policies for climate change and similar risks.
	  Vandenbergh and I made equal contributions to this paper. I analyzed the potential for catastrophic risk and the problems this posed for cost-benefit analysis. Vandenbergh built on this to explore the implications for policy analysis.
	  \end{credit}
%	
	\item
	\enquote{Energy and climate change: Key lessons for implementing the behavioral
	  wedge,}  by A.R. Carrico, M.P. Vandenbergh, P.C. Stern, G.T. Gardner,
	  T. Dietz, and \textbf{J.M. Gilligan}, Journal of Energy \& Environmental Law
	  \textbf{2}, 61--67 (2011). 
	  \begin{credit}
	  Applying social psychology to analyzing the obstacles to reducing individual and household greenhouse gas emissions, and identifying opportunities to surmount those obstacles. Ten psychological principles emerge as key factors that a successful policy must address.
	  I performed calculations and research on assessing the environmental impacts of individual actions. Carrico did most of the writing and psychological research. 
	  \end{credit}
%	
\iffalse
	\item
	\enquote{Implementing the behavioral wedge,}  by M.P. Vandenbergh, P.C.
	  Stern, G.T. Gardner, T.~Dietz, and \textbf{J.M. Gilligan}, Environmental Forum
	  \textbf{28}, 54--63 (Jul/Aug~2011). \textbf{\scshape Note:} \emph{Reprint of
	  Vandenbergh et al., Environ. L. Rep. \textbf{40}, 10\,547 (2010) as featured
	  cover story for summmer reading issue.}  
	  \begin{credit}
	  Identifying and debunking several myths about individual and household behavior as a source of greenhouse gas emissions and an analysis of the effectiveness of several recent government programs to promote energy efficiency. The principles we identified in previous work correlate well with the effectiveness of the policies analyzed here.
	  Vandenbergh did most of the writing and policy analysis, drawing on calculations and other research that I had performed for assessing the environmental impacts of individual actions to reduce energy consumption.
	  \end{credit}
\fi
%	
	\item
	\enquote{Implementing the behavioral wedge: Designing and adopting effective
	  carbon emissions reduction programs,}  by M.P. Vandenbergh, P.C. Stern,
	  G.T. Gardner, T.~Dietz, and \textbf{J.M. Gilligan}, Environmental Law Reporter
	  \textbf{40}, 10\,547--54 (2010).  \emph{Selected by Environmental Law Institute to reprint as the featured cover story of the 2010 summer reading issue for policymakers of Environmental Forum.}
	  \begin{credit}
	  Identifying and debunking several myths about individual and household behavior as a source of greenhouse gas emissions and an analysis of the effectiveness of several recent government programs to promote energy efficiency. The principles we identified in previous work correlate well with the effectiveness of the policies analyzed here.
	  Vandenbergh did most of the writing and policy analysis, drawing on calculations and other research that I had performed for assessing the environmental impacts of individual actions to reduce energy consumption.
	  \end{credit}
%	
	\item
	\enquote{Design principles for carbon emissions reduction programs,}  by P.C.
	  Stern, G.T. Gardner, M.P. Vandenbergh, T.~Dietz, and \textbf{J.M. Gilligan},
	  Environmental Science and Technology \textbf{44}, 4847--48 (2010).  
	  \begin{credit}
	  Analysis of best practices in past energy-conservation and efficiency programs reveals a set of six principles for effectively stimulating voluntary actions to reduce greenhouse gas emissions.
	  Stern did most of the writing drawing on calculations and other research that I had performed for assessing the environmental impacts of the measures he proposed.
	  \end{credit}
%	
	\item
	\enquote{The behavioral wedge: best policies to promote voluntary greenhouse
	  gas reductions by individuals and households,}  by \textbf{J.~Gilligan}, T.~Dietz,
	  G.~Gardner, P.~Stern, and M.~Vandenbergh, Significance \textbf{7}, 17--20
	  (2010). {\bfseries\scshape Note:} \emph{Invited paper, subsequently named one of the best papers of 2009 by Significance.}  
	  \begin{credit}
	  A summary of the behavioral wedge concept (Dietz \emph{et al.}, 2009) and an analysis of the statistical aspects and concerns regarding the analysis therein. I performed all the calculations and wrote the paper, drawing on the results of previous collaborative research with my co-authors.
	  \end{credit}
%	
	\item
	\textdagger\ 
	\enquote{Household actions can provide a behavioral wedge to rapidly reduce
	  {U.S.} carbon emissions,}  by T.~Dietz, G.~Gardner, \textbf{J.~Gilligan}, P.~Stern,
	  and M.~Vandenbergh, Proc.\ Nat'l.\ Acad.\ Sci.{} \textbf{106},
	  18\,452--18\,456 (2009).  
	  \begin{credit}
	  A collection of simple actions that would not require significant change in lifestyle could potentially reduce the greenhouse gas emissions from the average U.S. household by almost 40\%. Analysis of best-practices from actual energy-efficiency and conservation programs suggests that a well-designed national program could persuade the average household to reduce its greenhouse gas emissions by 20\% through purely voluntary actions. This would reduce total U.S. emissions by 7.4\%.
	  I performed all the calculations, did about half the assessments of the emissions reduction potential for different actions and the penetration of those actions in the contemporary population. I contributed substantially to writing the paper and wrote most of the supplementary information document.
	  \end{credit}
%	
	\item
	\textdagger\ 
	\enquote{The potential of dual camera systems for multimodal imaging of cardiac
	  electrophysiology and metabolism,}  by *M.R. Holcomb, *M.C. Woods, *I.~Uzelac,
	  J.P. Wikswo, \textbf{J.M. Gilligan}, and V.Y. Sidorov, Exper.\ Biol.\ \& Med.{}
	  \textbf{234}, 1355--1373 (2009). {\bfseries\scshape Note:} \emph{Selected by
	  the editors as the feature article of the month.}  
	  \begin{credit}
	  Design and characterization of a high-speed dual-camera system for research in electrocardiology, capable of simultaneously recording transmembrane electrical potential and calcium concentration across the entire surface of a beating heart.
	  I performed the original optical design and fabrication of the dual-camera system, wrote the high-speed image-acquisition software, and helped supervise Holcomb as he took over the project and brought it to fruition as his doctoral dissertation.
	  \end{credit}
%	
	\item
	\textdagger\ 
	\enquote{Costly myths: an analysis of idling beliefs and behavior in personal
	  motor vehicles,}  by *A.R. Carrico, *P.~Padgett, M.P. Vandenbergh,
	  \textbf{J.~Gilligan}, and K.A. Wallston, Energy Policy \textbf{37}, 2881--2888
	  (2009).  
	  \begin{credit}
	  Reducing unnecessary motor vehicle idling could reduce emissions of air pollution and greenhouse gases while saving considerable money by reducing gasoline consumption and wear and tear on the engine. However, a survey found that most people misunderstand this and believe that idling for short periods is good for the engine, saves money, and does not contribute significantly to pollution. Correcting these misperceptions could reduce pollution and greenhouse gas emissions while saving money.	  I conducted a literature review of motor vehicle idling, calculated the emissions resulting from idling, and the potential cost-savings and emissions reduction from reduced idling. I contributed to the statistical analysis of the survey data and to the writing of the paper.
	  \end{credit}
%	
	\item
	\enquote{Individual carbon emissions: The low-hanging fruit,}  by M.P.
	  Vandenbergh, J.~Barkenbus, and \textbf{J.M. Gilligan}, UCLA Law Review \textbf{55},
	  1701--1758 (2008).  
	  \begin{credit}
	  Analyzing the potential for a few voluntary actions by individuals to produce significant reductions of greenhouse gases while saving a great deal of money if they were widely adopted.
	  I performed roughly half of the quantitative analysis and one third of the writing.
	  \end{credit}
%	
	\item
	\textdagger\ 
	\enquote{A high-voltage cardiac stimulator for field shocks of a whole heart in
	  a bath,}  by *D.N. Mashburn, *S.~Hinkson, *M.C. Woods, \textbf{J.M. Gilligan}, *M.R.
	  Holcomb, and J.P. Wikswo, Rev. Sci. Instrum. \textbf{78}, 104\,302--104\,309
	  (2007).  
	  \begin{credit}
	  Design and characterization of a high-voltage stimulator for research in electrocardiology, capable of delivering a precise amount of electrical energy as a fast jolt under computer control.
	  I supervised Mashburn in designing and building the stimulator as an undergraduate research project and contributed to writing the paper.
	  \end{credit}
%	
	\item
	\enquote{Flexibility, clarity, and legitimacy: Considerations for managing
	  nanotechnolgy risks,}  by \textbf{J.M. Gilligan}, Environmental Law Reporter
	  \textbf{36}, 10\,924--10\,930 (2006).  
	  \begin{credit}
	  Policy analysis of health and environmental safety concerns relating to nanotechnology. 
	  This paper is entirely my work.
	  \end{credit}
%	
	\item
	\textdagger\ 
	\enquote{Time-resolved light scattering measurements of cartilage and cornea
	  denaturation due to free-electron laser radiation,}  by E.~Sobol,
	  A.~Sviridov, M.~Kitai, \textbf{J.M. Gilligan}, G.S. Edwards, and N.H. Tolk, J.
	  Biomed.\ Opt. \textbf{8}, 216--222 (2003). 
	  \begin{credit}
	  This paper reported on advances in understanding the detailed physical processes by which laser surgery works and to investigate the potential of laser-modification of tissue.
	  I contributed substantially to designing and conducting the experiments and measurements and to writing the paper.
	  \end{credit}
%	
	\item
	\textdagger\ 
	\enquote{Surface characterisation by near-field microscopy and atomic force microscopy,} by A. Cricenti, R. Generosi, M. Luce, P. Perfetti, G. Margaritondo, D. Talley, J.S. Sanghera, I.D. Aggarwal, \textbf{J.M. Gilligan}, and N.H. Tolk, Adv.\ Sci.\ Technol.\ \textbf{32}, 183--192 (2003).
	\begin{credit}
	This reported detailed studies of the chemical and electronic properties at the grain boundaries in a polycrystalline diamond film, demonstrating that the presence of hydrogen atoms at the grain boundaries could be mapped with very high spatial resolution. I contributed significantly to designing and performing the experiments.
	\end{credit}
%
	\item 
	\textdagger\ 
	\enquote{Defect transition energies and the density of electronic states in hydrogenated amorphous silicon,} by *G. Mensing, \textbf{J. Gilligan}, P. Hari, *E. Hurt, G. L\"upke, S. Pantelides, N. Tolk, and P.C. Taylor, J. Non-Cryst.\ Solids \textbf{299}, 621--625 (2002).
	\begin{credit}
	Amorphous silicon photovoltaic devices are much less expensive than single-crystal ones, but the Staebler-Wronski effect causes their performance to degrade quickly in sunlight. This paper reported on spectroscopic investigations of defects in amorphous silicon that might contribute to this effect. I supervised the graduate student designing and performing these measurements, performed extensive theoretical analysis and computer modeling of the defect states, and did a lot of the writing. 
	\end{credit}
%
	\item
	\textdagger\ 
	\enquote{Spectroscopic scanning near-field optical microscopy with a free
	  electron laser: {CH$_2$}~bond imaging in diamond films,}  by A.~Cricenti,
	  R.~Generosi, M.~Luce, P.~Perfetti, G.~Margaritondo, D.~Talley, J.~Sanghera,
	  I.~Aggarwal, \textbf{J.M. Gilligan}, and N.H. Tolk, J. Microsc. \textbf{202},
	  446--450 (2001). 
	  \begin{credit}
	  This paper reported on using a scanning near-field microscope with a free-electron laser to characterize the chemical and electronic properties of a polycrystalline diamond file with very high spatial resolution.
	  I participated in recruiting the team of researchers from several institutions, and contributed extensively to performing the experiments and analyzing data, and writing the paper.
	  \end{credit}
%
	\item
	\textdagger\ 
	\enquote{Scanning near-field infrared microscopy using chalcogenide fiber tips,} by D.B. Talley, L.B. Shaw, J.S. Sanghera, I.D. Aggarwal, A. Cricenti, R. Generosi, M. Luce, G. Margaritondo, \textbf{J.M. Gilligan}, and N.H. Tolk, Materials Lett. \textbf{42}, 339--344 (2000).
	\begin{credit}
	A review of the performance of chalcogenide fiber-optic probes for performing spectroscopy with near-field microscopic spatial resolution.
	I contributed significantly to designing and performing the experiments.
	\end{credit}
%	
	\item
	\textdagger\ 
	\enquote{Materials science at the WM Keck Free-Electron Laser: Infrared wavelength-selective materials modification,} by G. L\"upke, *C. Parks Cheney, *J. Sturman, *J.C. Keay, \textbf{J.M. Gilligan}, L.C. Feldman, and N.H. Tolk, Condensed Matter Theories \textbf{14}, 349--364 (2000).
	\begin{credit}
	A review of work by the research group on selectively modifying materials using a tunable mid-infrared free-electron laser.
	I designed and performed many of the experiments described here and conducted a substantial amount of the data analysis and interpretation.
	\end{credit}
%	
	\item
	\textdagger\ 
	\enquote{Chemical contrast observed at a {III-V} heterostructure by scanning
	  near-field optical microscopy,}  by A.~Cricenti, R.~Generosi, G.~Herold,
	  P.~Chiaradia, P.~Perfetti, G.~Margaritondo, \textbf{J.M. Gilligan}, and N.H. Tolk,
	  Phys.\ Stat.\ Solidi a-Appl.\ Res. \textbf{175}, 345--9 (1999). 
	  \begin{credit}
	  This paper reported on using a scanning near-field microscope with a free-electron laser to characterize the chemical and electronic properties of semiconductor interfaces with very high spatial resolution.
	  I participated in recruiting the team of researchers from several institutions, and contributed extensively to performing the experiments and analyzing data, and writing the paper.
	  \end{credit}
%	
	\item
	\textdagger\ 
	\enquote{Interface applications of scanning near-field optical microscopy with
	  a free electron laser,}  by A.~Cricenti, R.~Generosi, P.~Perfetti,
	  G.~Margaritondo, J.~Almeida, \textbf{J.M. Gilligan}, N.H. Tolk, C.~Coluzza,
	  M.~Spajer, D.~Courjon, and I.D. Aggarwal, Phys.\ Stat.\ Solidi a-Appl.\ Res.
	  \textbf{175}, 317--29 (1999). 
	  \begin{credit}
	  This paper reported on using a scanning near-field microscope with a free-electron laser to characterize the chemical and electronic properties of semiconductor interfaces with very high spatial resolution.
	  I participated in recruiting the team of researchers from several institutions, and contributed extensively to performing the experiments and analyzing data, and writing the paper.
	  \end{credit}
%	
	\item
	\textdagger\ 
	\enquote{Nonlinear energy-selective nanoscale modifications of materials and
	  dynamics in metals and semiconductors,}  by *S.~Marka, *C.P. Cheney, *W.~Wang,
	  G.~L\"upke, \textbf{J.~Gilligan}, *Y.~Yao, and N.H. Tolk, Sov.\ Phys.\ Tech.\ Phys.
	  \textbf{44}, 1069--72 (1999). 
	  \begin{credit}
	  This was a review of work by the research group on selectively modifying the chemical and physical properties of materials using tunable infrared lasers. I participated in performing the experiments and writing the paper.
	  \end{credit}
%	
	\item
	\textdagger\ 
	\enquote{Singlemode chalcogenide fiber infrared {SNOM} probes,}  by D.T.
	  Schaafsma, R.~Mossadegh, J.S. Sanghera, I.D. Aggarwal, \textbf{J.M. Gilligan},
	  N.H. Tolk, M.~Luce, R.~Generosi, P.~Perfetti, A.~Cricenti, and
	  G.~Margaritondo, Ultramicroscopy \textbf{77}, 77--81 (1999). 
	  \begin{credit}
	  This reported the characteristics of fiber-optic probes that made it possible to perform near-field microscopy with mid-infrared light from a free-electron laser.
	  This was primarily the work of Schaafsma and Aggarwal. I participated in recruiting the team of researchers and contributed to designing and performing the experiments in which the performance of the fiber tips was evaluated.
	  \end{credit}
%	
	\item
	\textdagger\ 
	\enquote{Fabrication of single-mode chalcogenide fiber probes for scanning
	  near-field infrared optical microscopy,}  by D.T. Schaafsma, R.~Mossadegh,
	  J.S. Sanghera, I.D. Aggarwal, M.~Luce, R.~Generosi, P.~Perfetti,
	  A.~Cricenti, \textbf{J.M. Gilligan}, and N.H. Tolk, Opt.\ Eng. \textbf{38}, 1381--5
	  (1999). 
	  \begin{credit}
	  This reported the fabrication of fiber-optic probes that made it possible to perform near-field microscopy with mid-infrared light from a free-electron laser.
	  This was primarily the work of Schaafsma and Aggarwal. I participated in recruiting the team of researchers and contributed to designing and performing the experiments in which the performance of the fiber tips was evaluated.
	  \end{credit}
	  
%	
	\item
	\textdagger\ 
	\enquote{First experimental results with the free electron laser coupled to a
	  scanning near-field optical microscope,}  by A.~Cricenti, R.~Generosi,
	  C.~Barchesi, M.~Luce, M.~Rinaldi, C.~Coluzza, P.~Perfetti, G.~Margaritondo,
	  D.T. Schaafsma, I.D. Aggarwal, \textbf{J.M. Gilligan}, and N.H. Tolk, Phys.\
	  Stat.\ Solidi a-Appl.\ Res. \textbf{170}, 241--7 (1998). 
	  \begin{credit}
	  This reported the first time a free-electron laser had been used to perform near-field infrared microscopy. Combining a free-electron laser with a near-field microscope allowed detailed studies of the chemical and electrical properties of materials with unprecedented spatial resolution (less than a micron). 
	  I participated in recruiting the team of researchers from several institutions, and contributed extensively to designing and performing the experiments and analyzing data, and writing the paper. I wrote software for visualizing the spectroscopy and surface topography simultaneously and analyzing the correlations between topography and spectroscopy.
	  \end{credit}
%	
	\item
	\textdagger\ 
	\enquote{Free-electron-laser near-field nanospectroscopy,}  by A.~Cricenti,
	  R.~Generosi, P.~Perfetti, \textbf{J.M. Gilligan}, N.H. Tolk, C.~Coluzza, and
	  G.~Margaritondo, Appl.\ Phys.\ Lett. \textbf{73}, 151--3 (1998). 
	  \begin{credit}
	  Combining a free-electron laser with a near-field microscope allowed detailed studies of the chemical and electrical properties of a polycrystalline diamond surface with unprecedented spatial resolution (less than a micron). 
	  I participated in recruiting the team of researchers from several institutions, and contributed extensively to designing and performing the experiments and analyzing data, and writing the paper. I wrote software for visualizing the spectroscopy and surface topography simultaneously and analyzing the correlations between topography and spectroscopy.
	  \end{credit}
%	
	\item
	\textdagger\ 
	\enquote{Coupled electron-hole dynamics at the {Si$/$SiO$_2$} interface,}  by
	  *W.~Wang, G.~L\"upke, M.~Di~Ventra, S.T. Pantelides, \textbf{J.M. Gilligan}, N.H.
	  Tolk, I.C. Kizilyalli, P.K. Roy, G.~Margaritondo, and G.~Lucovsky, Phys.\
	  Rev.\ Lett. \textbf{81}, 4224--7 (1998). This was principally Wang's and L\"upke's work. I contributed to data analysis and writing the paper.
	  \begin{credit}
	  This paper reported the observation of a novel phenomenon at semiconductor interfaces that can  significantly affect the performance of electronic devices. Understanding this effect has important implications for new-generation microelectronics.
	  The experimental work was principally by Wang and L\"upke. I contributed some ideas and discussion during the data analysis.
	  \end{credit}
%	
	\item
	\textdagger\ 
	\enquote{New molecular collisional interaction effect in low-energy
	  sputtering,}  by *Y.~Yao, *Z.~Hargitai, *M.~Albert, R.G. Albridge, A.V.
	  Barnes, \textbf{J.M. Gilligan}, *B.P. Ferguson, G.~L\"upke, *V.D. Gordon, N.H. Tolk,
	  J.C. Tully, G.~Betz, and W.~Husinsky, Phys.\ Rev.\ Lett. \textbf{81}, 550--3
	  (1998).
	  \begin{credit}
	  This reported on novel effects in which molecules striking gold surfaces at low energies removed many more gold atoms than atoms colliding at the same energies. This pointed to a key role for the internal energy of the molecules, which was surprising, since the internal energy is much smaller than the kinetic energy and had not been expected to make a noticeable difference. A theoretical model was developed to explain the effect.
	  This was principally Hartigai's and Yao's work. I contributed significantly to designing and performing the experiments and analyzing the data. Tully, Betz, and Husinsky developed the theoretical model.
	  \end{credit}
%	
	\item
	\textdagger\ 
	\enquote{Infrared wavelength-selective photodesorption on diamond surfaces,}
	  by *J.~Sturmann, R.G. Albridge, A.V. Barnes, J.L. Davidson, \textbf{J.M. Gilligan},
	  G.~L\"upke, *A.~Ueda, and N.H. Tolk, Appl.\ Surf.\ Sci. \textbf{129}, 59--63
	  (1998). 
	  \begin{credit}
	  This reported on the use of a tunable infrared free-electron laser to selectively remove hydrogen from a diamond surface. Selective desorption has potential as a technique for industrial semiconductor processing.
	  This was principally Sturmann's work. I contributed significantly to designing and performing the experiments.
	  \end{credit}
%	
	\item
	\textdagger\ 
	\enquote{Molecular effects in measured sputtering yields on gold at near
	  threshold energies,}  by N.H. Tolk, *Z.~Hargitai, *Y.~Yao, *B.~Pratt-Ferguson,
	  *M.M. Albert, R.G. Albridge, A.V. Barnes, \textbf{J.M. Gilligan}, *V.D. Gordon,
	  G.~L\"upke, A.~Puckett, J.~Tully, G.~Betz, and W.~Husinsky, Izv.\ Akad.\ Nauk.\
	  Ser.\ Fiz. \textbf{62}, 676--9 (1998). 
	  \begin{credit}
	  This reported on novel effects in which molecules striking gold surfaces at low energies removed many more gold atoms than atoms colliding at the same energies. This pointed to a key role for the internal energy of the molecules, which was surprising, since the internal energy is much smaller than the kinetic energy and had not been expected to make a noticeable difference. A theoretical model was developed to explain the effect.
	  This was principally Hartigai's and Yao's work. I contributed significantly to designing and performing the experiments and analyzing the data. Tully, Betz, and Husinsky developed the theoretical model.
	  \end{credit}
%	
	\item
	\textdagger\ 
	\enquote{Evaluation of source gas lifetimes from stratospheric observations,}
	  by *C.M. Volk, J.W. Elkins, D.W. Fahey, G.S. Dutton, \textbf{J.M. Gilligan},
	  M.~Loewenstein, J.R. Podolske, K.R. Chan, and M.R. Gunson, J. Geophys.\
	  Res.\ Atmos. \textbf{102}, 25\,543--64 (1997). 
	  \begin{credit}
	  This paper used \emph{in situ\/} measurements of a number of tracer compounds and ozone-depleting chemicals in the stratosphere to determine the atmospheric lifetimes of ozone-depleting substances. This allowed better estimates of the ozone depletion potential of these substances. Most of these measured lifetimes were shorter than previous estimates, thus implying that the ozone layer would recover more quickly than previously believed. This was principally Volk's work. I was team leader for the instrument that produced most of the data used in this paper. I supervised the design, construction, and operation of the instrument; contributed significantly to performing raw data analysis and contributed to writing the paper. Volk performed the detailed analysis and modeling. 
	  \end{credit}
%	
	\item
	\textdagger\ 
	\enquote{Photoexcitation spectroscopy and material alteration with
	  free-electron laser,}  by *J.~Sturmann, R.G. Albridge, A.V. Barnes,
	  \textbf{J.~Gilligan}, *M.T. Graham, J.T. McKinley, *W.~Wang, *X.~Yang, N.H. Tolk,
	  J.L. Davidson, and G.~Margaritondo, Act.\ Phys.\ Polon.\ A \textbf{91},
	  689--96 (1997).
	  \begin{credit}
	   This paper reviewed work by the research group on using infrared free-electron lasers to measure and selectively modify materials, such as diamonds and semiconductors.
	  I contributed significantly to designing and performing many of the experiments described.
	  \end{credit}
%	
	\item
	\textdagger\ 
	\enquote{Quantifying transport between the tropical and mid-latitude lower
	  stratosphere,}  by *C.M. Volk, J.W. Elkins, D.W. Fahey, R.J. Salawitch,
	  G.S. Dutton, \textbf{J.M. Gilligan}, M.H. Proffitt, M.~Loewenstein, J.R. Podolske,
	  K.~Minschwaner, J.J. Margitan, and K.R. Chan, Science \textbf{272}, 1763--8
	  (1996).
	  \begin{credit}
	  This paper used simultaneous \emph{in-situ\/} measurements of many tracer chemicals to determine that the transport of air from the tropics and middle latitudes into the lower mid-latitude stratosphere was significantly different than predicted by models, which implied that ozone was more vulnerable than predicted to chlorinated industrial chemicals and that stratospheric ozone was also vulnerable to emissions from supersonic aircraft.
	  This was principally Volk's work. I was team leader for the instrument that produced most of the data used in this paper. I supervised the design, construction, and operation of the instrument; contributed significantly to performing raw data analysis and contributed to writing the paper. Volk performed the detailed analysis and modeling. 
	  \end{credit}
%	
	\item
	\textdagger\ 
	\enquote{Airborne gas chromatograph for \emph{in situ\/} measurements of
	  long-lived species in the upper troposphere and lower stratosphere,}  by
	  J.W. Elkins, D.W. Fahey, \textbf{J.M. Gilligan}, G.S. Dutton, T.J. Baring, *C.M.
	  Volk, R.E. Dunn, R.C. Myers, S.A. Montzka, P.R. Wamsley, A.H. Hayden,
	  J.H. Butler, T.M. Thompson, T.H. Swanson, E.J. Dlugokencky, P.C.
	  Novelli, D.F. Hurst, J.M. Lobert, S.J. Ciciora, R.J. McLaughlin, T.L.
	  Thompson, R.H. Winkler, P.J. Fraser, L.P. Steele, and *M.P. Lucarelli,
	  Geophys.\ Res.\ Lett. \textbf{23}, 347--50 (1996). 
	  \begin{credit}
	  This paper reported on a novel four-channel gas chromatograph that could operate autonomously in a high-altitude high-performance aircraft to simultaneously measure concentrations of 10 different chemicals in the stratosphere, including the chlorinated organic compounds most responsible for ozone depletion.
	  I was the team-leader for the instrument described here. I supervised the design, construction, and operation of the instrument and contributed significantly to data analysis and writing the paper.
	  \end{credit}
%	
	\item
	\textdagger\ 
	\enquote{Estimates of total organic and inorganic chlorine in the lower
	  stratosphere from \emph{in situ\/} measurements during \textsc{aase ii},}  by
	  E.L. Woodbridge, J.W. Elkins, D.W. Fahey, L.E. Heidt, S.~Solomon, T.J.
	  Baring, T.J. Gilpin, W.H. Pollock, S.M. Schauffler, E.L. Atlas,
	  M.~Lowenstein, J.R. Podolske, C.R. Webster, R.D. May, \textbf{J.M. Gilligan},
	  S.A. Montzka, K.A. Boering, and R.J. Salawitch, J.\ Geophys.\ Res.
	  \textbf{100}, 3057--64 (1995). 
	  \begin{credit}
	  This reported a detailed inventory of the different sources of chlorine in the stratosphere through extensive measurements using airborne instruments. This assessment significantly improved the accuracy of calculations of predicted ozone loss.
	  I was the team leader for an airborne gas chromatograph that played a small part in producing the data for this paper. I contributed a small part to writing the paper.
	  \end{credit}
%	
	\item
	\textdagger\ 
	\enquote{Young's interference experiment with light scattered from two atoms,}
	  by U.~Eichmann, J.C. Bergquist, J.J. Bollinger, \textbf{J.M. Gilligan}, W.M.
	  Itano, D.J. Wineland, and M.G. Raizen, Phys.\ Rev.\ Lett. \textbf{70},
	  2359--62 (1993). 
	  \begin{credit}
	  This reported a novel quantum-optical effect. Young's two-slit experiment, in which light travelling through a pair of small slits exhibits interference, is well known, but here we demonstrated the effect using a single atom in place of each slit, and used a novel optical technique to demonstrate the quantum-mechanical \emph{welcher-weg} effect where the interference disappears when analysis of the light allows one to infer which slit (or atom) it passed through.
	  This work was principally Eichmann's. I invented the Brewster-plate measurement technique that made the \emph{welcher-weg} measurements possible. I contributed to performing the experiments and writing the paper.
	  \end{credit}
%	
	\item
	\textdagger\ 
	\enquote{{H$_2$}, {D$_2$}, and {HD} ionization potentials by accurate
	  calibration of several iodine lines,}  by D.~Shiner, \textbf{J.M. Gilligan}, *B.M.
	  Cook, and W.~Lichten, Phys.\ Rev.\ A \textbf{47}, 4042--5 (1993). 
	  \begin{credit}
	  This reported a precise calibration of spectroscopic references used in measuring the ionization potentials of the hydrogen molecule and its isotopic variants. These calibrations improved the precision of the previous measurements by 30--50\%. I conceived the experiment and contributed to performing the measurements, data analysis, and writing.
	  \end{credit}
%	
	\item
	\textdagger\ 
	\enquote{Quantum projection noise: Population fluctuations in two-level
	  systems,}  by W.M. Itano, J.C. Bergquist, J.J. Bollinger, \textbf{J.M. Gilligan},
	  F.L. Moore, and M.G. Raizen, Phys.\ Rev.\ A \textbf{47}, 3554--70 (1993). 
	  \begin{credit}
	  This paper reported for the first time the contribution of the intrinsic randomness of quantum-mechanical measurements to the uncertainty of precise spectroscopy, and its role as a fundamental limit to the precision of atomic clocks.
	  I participated significantly in performing all of the experiments and contributed to the data analysis and writing.
	  \end{credit}
%	
	\item
	\textdagger\ 
	\enquote{Ultra-high precision spectroscopy for fundamental physics,}  by W.M.
	  Itano, J.C. Bergquist, J.J. Bollinger, \textbf{J.M. Gilligan}, D.J. Heinzen, F.L.
	  Moore, M.G. Raizen, and D.J. Wineland, Hyperfine Interactions \textbf{78},
	  211--20 (1993).
	  \begin{credit}
	  This was a review of a large body of work by the research group.  
	  I performed high-precision measurements of hyperfine splitting of magnesium ions.
	  \end{credit}
%	
	\item
	\textdagger\ 
	\enquote{Quantum measurements of trapped ions,}  by W.M. Itano, J.C.
	  Bergquist, J.J. Bollinger, \textbf{J.M. Gilligan}, D.J. Heinzen, F.L. Moore, M.G.
	  Raizen, and D.J. Wineland, Vistas in Astronomy pp. 169--83 (1993). 
	  \begin{credit}
	  This was a review of a large body of work by the research group. I made significant contributions to measuring quantum projection noise and interference in light from two trapped ions.
	  \end{credit}
%	
	\item
	\textdagger\ 
	\enquote{Precise determinations of ionization potentials and {$EF$} state
	  energy levels of {H$_2$}, {HD}, and {D$_2$},}  by \textbf{J.M. Gilligan} and E.E.
	  Eyler, Phys.\ Rev.\ A \textbf{46}, 3676--90 (1992).
	  \begin{credit}
	  This paper reports on deep-ultraviolet spectroscopy of the isotopic variants of the hydrogen molecule with sufficient precision to allow the first rigorous tests of theoretical calculations of quantum-electrodynamic effects. I designed and performed all experiments, including building a large fraction of the apparatus, performed all the data analysis, and did almost all writing.
	  \end{credit}
%	
	\item
	\textdagger\ 
	\enquote{Ionic crystals in a linear {P}aul trap,}  by M.G. Raizen, \textbf{J.M.
	  Gilligan}, J.C. Bergquist, W.M. Itano, and D.J. Wineland, Phys.\ Rev.\ A
	  \textbf{45}, 6493--501 (1992). 
	  \begin{credit}
	  This paper reports on the first measurements of crystalline configurations of ions in a linear trap and demonstrates the suitability of this trap for a new generation of atomic clocks. Raizen designed and built the trap. I helped Raizen extensively with the experiments. I performed computer modeling of ionic crystals in the trap, and did about half the writing of this paper.
	  \end{credit}
%	
	\item
	\textdagger\ 
	\enquote{Linear trap for high-accuracy spectroscopy of stored ions,}  by M.G.
	  Raizen, \textbf{J.M. Gilligan}, J.C. Bergquist, W.M. Itano, and D.J. Wineland, J.\
	  Modern.\ Opt. \textbf{39}, 233--42 (1992). 
	  \begin{credit}
	  This paper reports a novel design for an ion trap in which a linear string of ions can be cooled to ultralow temperatures at which they form crystal structures suitable for a new generation of quantum-optics experiments. Raizen designed and built the trap. I helped Raizen extensively with the experiments. I performed computer modeling of ionic crystals in the trap, and did about half the writing of this paper.
	  \end{credit}
%	
	\item
	\textdagger\ 
	\enquote{High-resolution three-photon spectroscopy and multiphoton interference
	  in molecular hydrogen,}  by \textbf{J.M. Gilligan} and E.E. Eyler, Phys.\ Rev.\ A
	  \textbf{43}, 6406--9 (1991). 
	  \begin{credit}
	  This paper reports on an attempt to extend multiphoton spectroscopy further into the ultraviolet than had previously been possible using a new three-photon spectroscopic technique. The spectroscopy did not work as anticipated, due to a multiphoton interference phenomenon. This paper reports on the interference effect, presents a theoretical explanation, and presents a way to get around it and uses that technique to improve the precision of several spectroscopic measurements by a factor of two. I designed and performed all experiments and analysis and did almost all writing. Eyler and I contributed equally to interpreting the results.
	  \end{credit}
%	
	\item
	\textdagger\ 
	\enquote{Measurement of high Rydberg states and the ionization potential of
	  {H$_2$},}  by E.~McCormack, \textbf{J.M. Gilligan}, C.~Cornaggia, and E.E. Eyler,
	  Phys.\ Rev.\ A \textbf{39}, 2260--3 (1989). 
	  \begin{credit}
	  This paper reports significant advances in using multiple pulsed lasers to extend the frontiers of precise multiphoton spectroscopy farther in the vacuum-ultraviolet region, and uses these new techniques to measure the ionization potential of hydrogen with more than a three-fold improvement in precision. I contributed significantly to developing two-laser system for these experiments, the frequency tripling technique for producing deep ultraviolet wavelengths; contributed equally with McCormack to carrying out the experiments and analyzing spectra; and contributed to writing the paper.
	  \end{credit}
%	
\iffalse
	\item
	\textdagger\ 
	\enquote{Precise photodissociation and multiphoton spectroscopy of $\H_2$,  by E.F. McCormack, E.E. Eyler, and \textbf{J.M. Gilligan}, J. Opt. Soc. B Opt. Phys. \textbf{4}, 80 (1987).  
	  \begin{credit}
	  This paper reports set of a novel techniques for precise far-ultraviolet spectroscopy and uses them to perform precise spectroscopy of vacuum-ultraviolet transitions in molecular hydrogen. I performed experiments (equally with McCormack), and contributed significantly to data analysis and writing.
	  \end{credit} 
\fi
%	
	\item
	\textdagger\ 
	\enquote{Precise two-photon spectroscopy of {$E\leftarrow X^*$} intervals in
	  {H$_2$},}  by E.E. Eyler, \textbf{J.M. Gilligan}, E.~McCormack, A.~Nussenzweig, and
	  E.~Pollack, Phys.\ Rev.\ A \textbf{36}, 3486--89 (1987).  
	  \begin{credit}
	  This paper reports set of a novel techniques for precise far-ultraviolet spectroscopy and uses them to perform precise spectroscopy of vacuum-ultraviolet transitions in molecular hydrogen. I performed experiments (equally with McCormack), and contributed significantly to data analysis and writing.
	  \end{credit} 
%
\end{enumerate}


