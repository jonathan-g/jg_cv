\begin{rubric}{Teaching Highlights}%
\entry*[2015--]``Agent- and Individual-Based Computational Modeling'' \textsc{ees 4760/5760}, Developed new course on agent-based computational modeling with emphasis on emergent phenomena and applications in environmental science, ecology, economics, public health, and urban planning.
\entry*[2010--2014]``Water and Social Justice in Bangladesh'' \textsc{ees 390}, Developed team-taught transdisciplinary graduate capstone seminar (with Steven Goodbred and Brooke Ackerly) combining perspectives from natural sciences, engineering, social sciences, and humanities to study water resources and hazards in Bangladesh with focus on rivers, ground water, and coastal environments. The seminar includes interactions with students and faculty at Bangladeshi universities and field-work in Bangladesh.
\entry*[2008--]``Global Climate Change'' \textsc{ees 201}, New interdisciplinary course on climate change in earth's with a focus on integrating the science, economics, politics, and ethics of anthropogenic climate change so students leave with a broad perspective on the big picture of the ways different scholarly disciplines contributed to understanding climate change and possible responses to it.
\entry*[2008--] Supervising honors thesis in Medicine, Health, \& Society by Kelley Coffman on risk communication and environmental contamination at Oak Ridge National Laboratory.
\entry*[2008--] Advising interdisciplinary major in Environmental Economic Policy by Jeremy Doochin.
\entry*[2006--2008] ``Deep Geological Disposal of High-Level Radioactive Waste'' \textsc{ce 299}. Developed team-taught transdisciplinary graduate capstone seminar (with Jim Clarke and Calvin Miller) on disposal of nuclear waste, with a focus on the proposed repository at Yucca Mountain. The seminar combined sociological, ethical, psychological, political, engineering, and geological perspectives on the proposed repository and featured fieldwork in Nevada both to examine the geology and hydrology of the region and to interact with politicians, public officials, and community activists.
\entry*[2006--]Coordinator, ``Transdisciplinary Initiative on Environmental Systems'' graduate program, Vanderbilt University. Coordinated interdisciplinary courses and activities to bring together graduate students and faculty from humanities, social sciences, natural sciences, and engineering to study environmental issues.
\entry*[2005]''New Global Crisis: Energy and Water Resources in the 21st Century'' \textsc{hum161}, Vanderbilt University. Co-taught a multidisciplinary undergraduate course on the science, politics, and ethics of energy and water resources.
\entry*[2004--]``Earth and the Atmosphere,'' \textsc{ees108}, Vanderbilt University. Created new course covering the atmosphere both from the perspective of weather and climate and also as a component of the earth system. Special topics on how weather, pollution, and global change affect human society and how science, economics, and politics interact to manage these impacts. 
\entry*[2004--]``Science and Democracy,'' \textsc{ees115f} First-year writing seminar, Vanderbilt University. Created freshman seminar on what constitutes science, separating good science from junk science, and how questions of what constitutes good science play into contemporary political and legal disputes.
\entry*[2002--2003]Supervised senior undergraduate research project by Megan O'Grady, who subsequently won an NSF fellowship and attended graduate school at Harvard.
\entry*[2000--2003]``Nonlinear Dynamics and Chaos,'' \textsc{phys361}, Vanderbilt University. Created new course on chaos and nonlinear dynamics in the natural sciences.
\entry*[2000]``Science in a Democracy,'' \textsc{hons189.02}, Vanderbilt University (with L. Branscomb).
\entry*[1997--2003]Dissertation committees: Served on five dissertation committees in Physics. Currently on committees for Luis Fong and Andreas Werdich.
\entry*[1996--]``Science, Risk, and Policy,'' \textsc{sth205} (later, \textsc{phys205}, \textsc{geol205}) Vanderbilt University (with B. Walter). Created interdisciplinary course on how society manages lethal risks.
\entry*[1995--1998]``Atmospheric Physics,'' \textsc{phys108}, Vanderbilt University. Revitalized atmospheric physics course with focus on the science, economics, and politics of global environmental change.
\end{rubric}
