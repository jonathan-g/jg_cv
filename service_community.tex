\item[2023] Featured speaker, ``An Opening Goodbye,'' a chamber-music concert
  by the ensemble Chatterbird, co-sponsored by Vanderbilt's Curb Ce3nter for
  Art, Enterprise, and Public Policy. The concert featured four new pieces by
  Vanderbilt faculty, each addressing an issue of contemporary social and political
  concern and paired with remarks from writers, political activists, and scholars.
  I commented on Professor Molly Herron's piece, ``An Opening Goodbye,''
  which engaged musically with climate change.  April 12.
\item[2022] Organized two-day workshop on using sustainable infrastructure to
  address urban-rural disparities in the Southeast. 50 participants represented
  government, universities, and private sector, including City of Nashville,
  Tennessee Valley Authority, AT\&T, Greater Nashville Regional Council,
  and Atlanta Regional Commission, Electric Power Research Institute,
  University of Tennessee Knoxville, Tennessee Tech, and Georgia Tech.
  Mar.~24--25.
\item[2020--2022] Advised Nashville Mayor's Sustainability Advisory
Committee on quantitative analysis of climate actions plans for the City of
Nashville.
\item[2020--2022] Keynote presentations on climate science for the Nashville
Youth Climate Summit. Feb.~8, 2020, Feb.~20, 2021, Mar.~5, 2022.
\item[2021] Panelist, 2021 Nashville Climate Summit, a webinar organized by
 AllianceBernstein to inform the Nashville business community about
 challenges and opportunities associated with climate change. April 21, 2021.
\item[2019] Presented tutorial on ``What Science Can and Cannot Say about
  Climate Change'' as part of a training workshop for journalists from the
  Southeastern U.S., organized by Inside Climate News at the Freedom Forum
  First Amendment Center, Nashville TN, Sept.~16.
\item[2019] Michael Vandenbergh and I briefed a team of 13 representatives of
  the Office of the Inspector General for the U.S. Environmental Protection
  Agency about private environmental governance relating to toxic substances.
  Nashville, TN, July~10.
\item[2019] Presented tutorial on ``What Science Can and Cannot Say about
  Climate Change'' as part of a training workshop for journalists from the
  Midwestern U.S., organized by Inside Climate News at the Freedom Forum First
  Amendment Center, Nashville TN, Mar.~7.
% \item[2018]  Met with newly-elected U.S. Representative Mark Green (TN-7) to answer his questions about climate science. Nashville TN, Dec.\ 17.
\item[2018] Addressed Rotary Club of McMinnville TN on the impacts of climate
  change in middle Tennessee. McMinnville TN, Dec.\ 6.
\item[2018] Intervewed by WCPI, McMinnville TN Public Radio station on the
  impacts of climate change in middle Tennessee. McMinnville TN, Dec.\ 6.
\item[2018] Addressed Breakfast Club of Nashville (businesswomen's group) on
  private-sector responses to climate change. Nashville TN, Nov.\ 29.
\item[2018] Presented tutorial on ``What Science Can and Cannot Say about
  Climate Change'' as part of a training workshop for journalists from the
  Southeastern U.S., organized by Inside Climate News at the Freedom Forum
  First Amendment Center, Nashville TN, Sept.~24.
\item[2018] Organized day-long workshop on ``Data-Methods for Equitable
  Development in Nashville,'' with participants from Metro Nashville
  government, Metropolitan Planning Organization,
  and many community groups.
\item[2017--2019] Member, Environmental Public Health Community Advisory
  Group, Metro Nashville Department of Public Health. Worked with Dr. Sanmi
  Areola (Deputy Director, Metro Department of Public Health)
%, Erin Jackson (Metro Public Health), Kimberly Jackson (Meharry Medical College and Health Impacts of Degraded Environments), Prof.\ Yolanda McDonald (Vanderbilt Community Action Program), and Carol Ziegler (Vanderbilt School of Nursing)
  to establish a research network for monitoring air quality in public housing
  units and provide research opportunities for Vanderbilt undergraduate and
  graduate students.
\item[2017] ``Beyond Gridlock: The Private Governance Response to Climate
  Change.'' Public lecture (with Michael Vandenbergh) at Nashville Public
  Library as part of the ``Thinking out of the Lunch Box'' series. (Apr. 5).
\item[2015--2017] Collaboration with University School of Nashville physics
  teacher Wilson Hubbell to incorporate scientific literacy about mathematical
  and computational modeling into high-school science curricula (Funding for
  USN from an Edwin E. Ford Leadership Challenge Grant).
\item[2011--2012] Co-author, \emph{Sustainable Tennessee}, a report to state
  and local decision makers on the impacts of climate change on Tennessee and
  possible adaptations. Oak Ridge National Laboratory and Sustainable Tennessee.
% \item[2012] Co-author, \textit{Sustainable Tennessee}, a report to state and local decision-makers on the impacts of climate change on Tennessee and possible adaptations.
\item[2009] Briefed representatives of Senators Corker and Alexander on
  environmental aspects of the Convention on the Law of the Sea Treaty
  (organized by the Pew Charitable Trusts), Nov.~18.
\item[2009] Invited panelist, ``Health in Tennessee: The Impact of Climate
  Change,'' Public Policy Forum with Tennessee State Legislature (organized by
  Papasan Institute for Government Relations, U. Memphis), June~3.
\item[2007--2009] Advisory Board on Environment, The Tennessean Newspaper.
\item[2008] Testimony about climate change before Tennessee House Committee on
  Conservation and Environment, Feb. 28.
\item[2006] Co-Organizer, Nashville Forum on Christianity and the Environment,
  Scarritt-Bennett Center, Sept.~30.
\item[2006] Panelist, Belcourt Theatre discussion of genetically modified food.
  Apr.~7.
\item[2005] ``Democracy in the Age of Science'' Public lecture at Nashville
  Public Library as part of the ``Thinking out of the Lunch Box'' series.
  (Sept. 7).
% \item[Ongoing] Regularly answer questions from local news media (Nashville Scene, The Tennessean, TV news, etc.) about environmental science and policy.
% \item[Ongoing] Regularly speak on environmental issues to local churches, community groups, and colleges.
