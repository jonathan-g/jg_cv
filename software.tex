%
% !TeX root = Gilligan_CV_2015.tex
%
\begin{enumerate}
%
\item 
kayadata: Kaya Identity Data for Nations and Regions
by \textbf{J.M. Gilligan}, 
Comprehensive R Archive Network (2019): \url{https://cran.r-project.org/web/packages/kayadata/}
%
\item 
kayatool: Interactive Energy and Emissions Policy Analysis Tool
by \textbf{J.M. Gilligan}, 
GitHub (2019): \url{https://github.com/jonathan-g/kayatool}
%
\item 
analyzeBehaviorspace: Interactive Analysis of Output from NetLogo Behaviorspace Experiments
by \textbf{J.M. Gilligan}
GitHub (2018): \url{https://github.com/jonathan-g/analyzeBehaviorspace}
%        
    \item 
    forecastVeg: Forecasting Vegetation Health  at High Spatial Resolution,
	    by J.J. Nay*, E.K. Burchfield*, and \textbf{J.M. Gilligan}, 
	    GitHub (2016): \url{https://github.com/JohnNay/forecastVeg}
\begin{credit}
I suggested the line of research that led to this software. The code is entirely Nay's and Burchfield's work. 
This open-source package automates downloading high-resolution MODIS spectral data, applying machine learning (gradient-boosted machines) to 
identifying patterns in the data, generating a forecasting model, and assessing the out-of-sample predictive skill using cross-validation. 
This tool is designed to facilitate the use of machine learning and satellite remote sensing data for decision-support
around the world, including in developing nations that do not have large research budgets.
\end{credit}
%
\item predMarket: Agent-based model of trader behavior in a climate prediction market.
by J.J. Nay*, M. Van der Linden*, and \textbf{J.M. Gilligan},
GitHub (2016): \url{https://github.com/jonathan-g/predMarket}
\begin{credit}
    This software model was originated by Nay and Van der Linden based on a paper by Vandenbergh, Raimi, 
    and myself.
    Van der Linden wrote code for implementing a simultaneous double-auction trading scheme.
    I wrote the climate prediction code and the code for initializing the social network of the traders.
    Nay and I wrote the code to run and analyze experiments with the model.
\end{credit}
%
\item 
datafsm: Estimating Finite State Machine Models from Data
by John J. Nay*, and \textbf{J.M. Gilligan}, 
Comprehensive R Archive Network (2015): \url{https://cran.r-project.org/web/packages/datafsm/}
\begin{credit}
    This software package was largely Nay's work. He had the original idea and did most of the programming. I contributed significantly to the design by suggesting that finite state machines and genetic algorithms would be the best way to implement Nay's concept. We shared equally in writing the ``Introduction to datafsm'' manual.
\end{credit}
%
\item
Floodpartsim: A Participatory Agent-Based Simulation of Urban Flood Risk Management
by \textbf{J.M. Gilligan}, C.E. Brady, J.V. Camp, J.J. Nay*, and P. Sengupta,
GitHub (2015): \url{https://github.com/jonathan-g/Floodpartsim}
\begin{credit}
    Sengupta and I developed the conceptual design and specifications of the model. Camp contributed expertise on hydrological modeling.
    Brady and Nay wrote most of the code.
\end{credit}
%    
\end{enumerate}


