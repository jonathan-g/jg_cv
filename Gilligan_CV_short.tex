% !TeX document-id = {d206223e-80cd-48b7-93a5-45cd17290331}
% !TeX program = lualatex
% !BIB program = biber
\NeedsTeXFormat{LaTeX2e}
\documentclass[10pt]{article}
\newif\ifcredit
\creditfalse

% \usepackage{jglucida}
\usepackage[LGR,LY1]{fontenc}
\usepackage[osf]{noto}
\usepackage{pifont}
%\setmonofont{Latin Modern Mono}
%\usepackage[ttdefault=true]{AnonymousPro}
%\setmonofont{Anonymous Pro}[Scale=MatchUppercase, ScaleAgain=0.9]
%\usepackage{inconsolata}
%\setmonofont{InconsolataN}[Scale=MatchLowercase]
\usepackage[bold-style=ISO]{unicode-math}
\usepackage{fontspec}

\setmathfont{Cambria Math}

\usepackage{microtype}

\newcommand{\thedate}{June 28, 2021}
\newcommand{\fullname}{Jonathan M. Gilligan}

\usepackage[%
  hidelinks,%
  breaklinks=true,%
  pdfpagemode=UseOutlines,%
  pdfstartview=Fit,%
  pdfpagelayout=SinglePage,%
  pdfpagelabels=true,%
  pdfusetitle=true,%
  ]{hyperref}
\usepackage{titling}
\usepackage{titlesec}
\usepackage{lastpage}
\usepackage{geometry}
\usepackage{fancyhdr}
\ifcredit
\usepackage[backend=biber,style=jgcv,credit=true]{biblatex}
\else
\usepackage[backend=biber,style=jgcv,credit=false]{biblatex}
\fi
\usepackage[version=4]{mhchem}
%\usepackage[squaren,Gray]{SIunits}
\usepackage{csquotes}
\usepackage[american]{babel}
\usepackage{comment}
%
%

\newcommand*{\Note}{\textbf{\scshape Note:}}
\renewcommand*{\peerrevmark}{\ding{52}}
%
%
\geometry{margin=1.0in}
%
%\renewcommand{\thesection}{\Alph{section}.}
%\renewcommand{\thesubsection}{\thesection \arabic{subsection}.}
%\renewcommand{\thesubsubsection}{\thesubsection \alph{subsubsection}.}
%\renewcommand{\thesection}{\Alph{section}.}
\renewcommand{\thesubsection}{\thesection \alph{subsection}.}
\renewcommand{\thesubsubsection}{\thesubsection \roman{subsubsection}.}
%
\titleformat{\section}{\large\sffamily\bfseries}{\thesection}{0.5em}{}
\titlespacing*{\section}{0pt}{2ex}{1ex}
%
\titleformat{\subsection}{\large\sffamily}{\thesubsection}{0.5em}{}
\titlespacing*{\subsection}{1em}{2ex}{1ex}
%
\titleformat{\subsubsection}{\sffamily\bfseries}{\thesubsubsection}{0.5em}{}
\titlespacing*{\subsubsection}{1.5em}{1.5ex}{0.5ex}
%
%
\hfuzz 5pt
%
%
% \newcommand{\enquote}[1]{``#1''}
\ifcredit
\newenvironment{credit}{\begin{quote}\itshape}{\end{quote}}
\else
\excludecomment{credit}
\fi
%
\pagestyle{fancy}

\renewcommand{\sectionmark}[1]{\markright{#1}}
\newcommand{\chaptermark}[1]{\markboth{#1}{}}

\fancyhead[L]{\sffamily\scshape \fullname}
\fancyfoot[R]{\sffamily\scshape \rightmark}
\fancyfoot[C]{\sffamily--\scshape\thepage\ of \pageref{LastPage}--}%
\fancyhead[R]{\sffamily\scshape\thedate}
%
\fancypagestyle{plain}{%
\fancyhf{} % clear all header and footer fields
\fancyfoot[C]{\sffamily--\scshape\thepage\ of \pageref{LastPage}--} % except the center
\renewcommand{\headrulewidth}{0pt}
\renewcommand{\footrulewidth}{0pt}}
%
\pretitle{\begin{center}\Large\sffamily\bfseries}
\posttitle{\par\end{center}\vskip 0.5em}
\preauthor{\begin{center}\large\lineskip 0.5em}
\postauthor{\par\end{center}}
\predate{\begin{center}}
\postdate{\par\end{center}}
\setlength{\droptitle}{-60pt}


\fancyfoot[R]{}
\addbibresource{jgpubs.bib}

%
\title{Short Curriculum Vitae}
\date{\thedate}
\author{Jonathan Mark Gilligan}
\begin{document}
\maketitle

\section{Contact Information}
\include*{contact}

\section{Degrees Earned}
\begin{description}
	\include*{degrees}
\end{description}
\section{Employment History}
\include*{employment}

\section{Honors and Awards}
\begin{description}
\item[2018] The Chancellor's Award for Research, Vanderbilt University
  (jointly with Michael Vandenbergh).
\item[2017] The Morrison Prize for the highest impact paper of the year
  on sustainability law and policy (jointly with Michael Vandenbergh).
\item[1998] Outstanding Scientific Paper Award, NOAA Environmental Research
  Labs.
\item[1995] NASA Group Achievement Award for outstanding accomplishments and
  contributions to the Airborne Southern Hemisphere Ozone Experiment and
  Measurements to Assess the Effects of Stratospheric Aircraft.
\item[1991--1993] National Research Council Postdoctoral Associate
\item[1985--1986] J.W. Gibbs Fellow, Yale University
\end{description}

\section{Summary of Published Scholarship}
    One book, 75~journal papers, 3~book chapters, 18~substantive papers in
    conference proceedings, 2~patents, 7~open-source software packages.
    \subsection{Citations and H-Index}
    \include*{citation-metrics}
	%
  \subsection[Selected Publications]{Selected Publications (* denotes student author, \peerrevmark\ denotes peer-reviewed)}
    \begin{refsection}
    \nocite{vandenbergh:2017:beyond.politics,carrico:marriage:2020,elsawah:2020:grand.challenges,gilligan:2020:assessing.private.governance,gilligan:2020:beyond.wickedness,nielsen:2020:mitigation.analysis,gilligan:2019:collaboration,gilligan:2018:water.conservation,gilligan:2018:carrots.sticks,nay:2018:vegetation.health,wilson:2017:infilling,burchfield:2016:ag.adaptation,gunda:2016:drought,auerbach:2015:polders,gilligan:2014:political.feasibility,dietz:2009:behavioral.wedge}
    \printbibliography[heading=none]
    \end{refsection}
	%
%	\subsection{Invited Presentations}
%  40 invited presentations.

	\section{Research Grants}
	PI or Co-PI on current and past grants totaling \$14 million since 2011.
	%
 \section{Selected Synergistic Activities}
 \begin{description}
   \item [2021--present] Associate Editor for Climate Law and Policy,
     \emph{Frontiers in Climate}.
   \item[2020--present] Director, Vanderbilt Climate and Society Grand
     Challenge Initiative.
  \item[2020--present] Advisor to Nashville Mayor's Sustainability Advisory
    Committee.
  \item[2020--present] External Advisory Committee, "Water Unaffordability in
    the United States," NSF-sponsored project. PI: Laura Senier (Northeastern U.).
    PI.
   \item[2016--present] External Advisory Committee,
     Urban Water Innovation Network, an NSF-sponsored sustainability research
     network with \$12.5 million in funding.
  \item[2016--present] Organizing Committee, Annual Conference on Artificial
    Intelligence and the Law, Vanderbilt Law School.
  \item[2016--present] Program Committee, Environmental and Sustainability
    Applications track, Winter Simulation Conference, co-sponsored by IEEE and
    INFORMS.
  \item[2018--19] Collaborated with Inside Climate News and the First Amendment
    Center to run training workshops about reporting on climate change for
    journalists in the Southeast and Midwest.
  \item[2018] Working Group on the Use of Socio-Environmental Systems Modeling
    in Actionable Science, National Socio-Environmental Synthesis Center.
 \end{description}
\end{document}
%% cv.tex ends here.
