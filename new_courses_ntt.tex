%
% !TeX root = Gilligan_CV.tex
%
\item ``Water and Social Justice in Bangladesh'' \textsc{ees 390}. Introduced Spring 2010. Developed team-taught transdisciplinary graduate capstone seminar (with Steven Goodbred and Brooke Ackerly) combining perspectives from natural sciences, engineering, social sciences, and humanities to study water resources and hazards in Bangladesh with focus on rivers, ground water, and coastal environments. The seminar includes interactions with students and faculty at Bangladeshi universities and field-work in Bangladesh.
\item ``Global Climate Change'' \textsc{ees 2110/5110}. Introduced Fall 2008. New interdisciplinary course on climate change in earth's with a focus on integrating the science, economics, politics, and ethics of anthropogenic climate change so students leave with a broad perspective on the big picture of the ways different scholarly disciplines contributed to understanding climate change and possible responses to it.
\item ``Science, Risk, and Policy,'' \textsc{ees2150} (formerly \textsc{ees 205}, \textsc{geol 205}). Introduced Spring 2004. Created interdisciplinary course on how society manages deadly risks.
\item ``Science and Democracy,'' \textsc{ees1111} (formerly \textsc{ees115f}). Introduced Fall 2004.  First-year writing seminar on what constitutes science, separating good science from junk science, and how questions of what constitutes good science play into contemporary political and legal disputes.
\item ``Deep Geological Disposal of High-Level Radioactive Waste'' \textsc{ce 299}. Introduced Spring 2007. Developed team-taught transdisciplinary graduate capstone seminar (with Jim Clarke and Calvin Miller) on disposal of nuclear waste, with a focus on the proposed repository at Yucca Mountain. The seminar combined sociological, ethical, psychological, political, engineering, and geological perspectives on the proposed respository and featured fieldwork in Nevada both to examine the geology and hydrology of the region and to interact with politicians, public officials, and community activists.
\item ''New Global Crisis: Energy and Water Resources in the 21st Century'' \textsc{hum161} (with David Furbish). Co-taught a multidisciplinary undergraduate course on the science, politics, and ethics of energy and water resources.
\item ``Earth and the Atmosphere,'' \textsc{ees108}. Introduced Spring 2004. The atmosphere from the perspective of weather and climate and also as a component of the earth system. Special topics on how weather, pollution, and global change affect human society and how science, economics, and politics interact to manage these impacts.
\item ``Nonlinear Dynamics and Chaos,'' \textsc{phys361}. Introduced Fall 2000. Developed a graduate seminar on nonlinear dynamics and chaos with emphasis on drawing connections between the formal mathematical foundations and applications to laboratory science and students' research.
\item ``Science in a Democracy,'' \textsc{hons189.02} (team-taught with Lewis Branscomb), Spring, 2000. Developed and co-taught an honors seminar on the interactions of science with public policy, examining issues of fraud and integrity in research, intellectual property, science as an engine of economic growth, technocracy vs.\ democracy, and environmental regulation. Featured guest lectures and class discussions with Senators Lamar Alexander and William Frist.
% \item ``Atmospheric Physics,'' \textsc{phys108}. Introduced Spring 1995. Introduced atmospheric physics course with focus on the science, economics, and politics of global environmental change.
